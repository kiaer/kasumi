\chapter{Discussion}
\label{ch:disc}

As most improvements available to us lies in the online phase, this
is mostly what will be discussed. Our biggest bottleneck when running
the online phase lies in the table lookups required to perform the
attack. As this is the case, we will have a look on improvements to
lookups and how to reduce them.

We will also look into using GPUs as a cheaper alternative to CPUs
core-for-core. Over the last decade the GPU has improved immensely and
is now a very viable alternative when looking at parallelization. In
the GPU part we will assume the CUDA will be able to make improvements
both the pre-computational phase and online phase.

Lastly we will try and discuss a possible hardware implementation of the attack
and benefits of such.

\section{DP/Rainbow}

\cite{nohl} proposed an alternative to the original Rainbow attack
method. Instead of using the original approach, they combined the DP
attack and rainbow attack thus resulting in a more efficient
TMTO-attack doing fewer operations with the bottleneck of the
harddrive. 

The idea proposes chain computations with one reduction function
until a distinguished point is reached, then changing the reduction
function for the next chain computation. This allows for a improvement
in lookups as the generated table will only require a lookup when a
distinguished point is reached in the online phase. 

At the same time the benefits of a rainbow attack is still maintained,
as no loops or merged chains and the low amount of tables is still
present. This is due to the combined approach still using different
multiple different reduction functions.

\section{Storage improvements}

As the tables in the attack grows larger, we have to make improvements
to reading/writing the files. In our case the tables of the full size
attack reaches a size of around \code{8TB}. From our tests we can
clearly see that the biggest bottleneck of the attack is the lookup
time. Knowing this we have looked into possibilities for improving the
lookup times required.

\subsection{Binary Search}

Since the binary search algorithm essentially finds an element in a
sorted array, we could use this algorithm for lookup in our
online phase. We will indeed have to change some major points of our
implementation. The biggest change will be a change to the sequential
approach. First off, since the binary search algorithm requires a
sorted list, we need to sort the table by end points. This also means
we need to store both start points and end points.

This has two implications on the pre-computational phase. Firstly the table
needs sorting and on larger tables this can be time consuming. This
can however be seen as a one-time operation that will greatly increase
the efficiency in the online phase. Secondly when going from the
sequential approach, where we can save memory, we have to 
increase the memory by a factor 2. This means that we either have to
generate new parameters if \code{8TB} of data is still the limit or
rethink our cost analysis for \code{16TB} of data.

A increase of factor 2 memory consumption can be seen as a small
increase compared to the increase in performance when doing a
lookup. The binary search algorithm has a worst case running time of
$O(log_2(N)$, which when our parameters taken into consideration will
have the following maximal amount of comparisons
\[m  = 2^{40} \Rightarrow Comparisons = log_2(2^{40}) = 40\]
\subsection{Hash map}
% The index table method can be seen as a special case of a more general and widely known data structure
% called hash tables. To store m starting point and ending point pairs,
% one first fixes a hash function that maps elements of N to log m-bit
% strings. This function need not be a cryptographic hash function,
% although the same term is used. Instead of sorting the data, each starting point
% and ending point pair is recorded at the position in the storage
% addressed by the hash value of the ending point. Collisions of
% addresses are inevitable, but there are various ways to deal with this
% problem. Table lookups to hash tables are performed by first hashing
% the ending point to be searched for in the table and fetching the data
% located at the address pointed to by the hash value. Since the address
% holds log m bits of information, even if almost log m bits from each
% ending point are removed before storage, we can reliably determine
% whether or not a match has occurred. One advantage of the hash table
% method, other than reducing storage and not requiring any sorting, is
% that it provides constant time table lookups. In comparison, a lookup
% to a sorted table requires time that is logarithmic in the table
% size. If the hash function is set to return the first ${(log m) - \epsilon }$
% bits of its input and buckets to hold approximately 2 $\epsilon$ table entries
% are placed at the position pointed to by each hash value, then the
% hash table technique reduces to the index table technique.

This is discussed in \cite[Appendix D]{176}.

TODO Skal vi explaine? Orker det næsten ikke...

\section{GPU Parallel/CUDA}

Modern day GPUs allow for parallelization with great effect. Core for
core the general purpose processor might be more powerful than a GPU.
But where a modern day CPU generally consists of 4-8 cores, modern
GPUs consists of 1000s of cores optimized for parallel
computations. The pre-computational phase of a TMTO-attack requires a
huge amount of similar computations, and this is where GPU
parallelization really shines. All operations in a table generation
are relatively simple, but requires huge amounts of repetitive work. 

We chose to have a deeper look into NVIDIAs
CUDA\footnote{\url{http://www.nvidia.com/object/cuda_home_new.html}}
as CUDA programming can be done using modified C. 

Looking at the high-end consumer graphics cards available at the
moment, we can take a look at the NVIDIA Titan X
card\footnote{\url{http://www.geforce.com/hardware/desktop-gpus/geforce-gtx-titan-x/specifications}}. This
card specifications includes 3072 CUDA cores running at a base clock
speed of \code{1000MHz}. This comes at a price of around
$1299.99\$$. We can compare this to the, at the moment, top model of Intel
proccesors Intel
i7-5950HQ\footnote{\url{http://ark.intel.com/products/87720/Intel-Core-i7-5950HQ-Processor-6M-Cache-up-to-3_80-GHz}}
which comes with 4 cores running at the base frequency of
\code{2.9GHz} with a price around $630\$$.

If we assume all cores perform at a maximal frequency at all times the
Titan X should be able to perform $3072 \cdot 10^9$ cycles each
second. For the Intel counter part, assuming the same, the CPU will
only be able to perform $11,6 \cdot 10^9$ cycles each second. The GPU
can therefore increase the cycles performed of around 250 times when
these assumptions is made. This is all done with a price increase of
only around the double. Looking at the performance with these
assumptions will undoubtedly not be realistic in a real-world
scenario. As examples
\begin{itemize}
\item Write/read from a disc is not taken into
consideration and could be a huge bottleneck. 
\item In a real-world case we can't assume that all cores are used
  with their maximal frequency at all times. Especially not when
  looking at CUDA cores.
\end{itemize}
As the cost of these high-end graphics cards is low compared to
the computational power gained, this is definitely a way worth considering.

\section{Hardware implementation}

The hardware implementation will mostly be directed at the
pre-computational phase. We still see it as a goal to run the online
phase on different setups and even cross platform. This is for the
most part speculations and comparison to previous work implemented as
hardware instead of software. We have not tried to test any part of
our implementation as hardware.

The first improvement that can be done looking at a hardware
implementation, is implementing a hardware solution for
KASUMI. Looking at \cite{TODO CITE HARDWARE PAPER} TODO, and the
performance levels they reached, it can be seen that a huge increase
in performance can be gained by running the KASUMI algorithm on
dedicated hardware. A much lower cycle count is needed for encryptions
and can be done with a much lower clock frequency. Since most of the
computations done in KASUMI is bit manipulation, this can also be done
with the mostly standard logic gates. 

Now thinking about the pre-computation of the table required to
perform the attack, we could also see a hardware implementation being
a possibility. Since the table generation is a very specific task with
a lot of repetitive work, hardware implementation could be ideal. With
an already much faster hardware implementation of KASUMI to take
advantage of, table generation could benefit from hardwares faster
execution of repetitive tasks. A hardware solution could also reduce
the cost of generating the tables. A rig consisting of chips designed for one specific
purpose will often be cheaper to buy than the costs of similar power
from a general purpose processor.


%%% Local Variables:
%%% mode: latex
%%% TeX-master: "Thesis"
%%% End:
