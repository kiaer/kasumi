\chapter{Introduction}

\section{KASUMI Cipher and 2G/3G}

In this project, we analyse the real-world security of KASUMI as used
in 2G and 3G networks in form of A5/3 towards time-memory trade-off
attacks. KASUMI is a block cipher algorithm used in mobile embedded
systems to provide security between the phone and the base station. As
GSM is the international standard for mobile communication, KASUMI
is subject to a lot of cryptanalysis.

\section{TMTO Attack}

We will discuss some of the most popular time-memory trade-offs, give
the background theory behind them and try and analyze the best
TMTO-attack to use with KASUMI in mind.

The  TMTO-attack will be analyzed and optimal parameters for an
actual attack will be discussed.


\section{The problem}

The project will require an optimized implementation of the KASUMI
block cipher as described in papers. With the KASUMI cipher
implemented, the TMTO attack can be implemented

As the KASUMI cipher uses a keys with the size of \code{64-bit}, we
will have trouble testing an attack with hardware at our disposal. As
a solution we will implement an experimental attack using a key with
size \code{32-bit}. The experimental attack will allow us to generate
test results and provide us with the possibility of estimating how the
attack on a full keysize would operate.

Knowing some of the previous attacks performed on KASUMI, we can in
the conclusion conclude whether or not this attack can be seen as practical.

\section{Structure of the Thesis}

First we will  go through the theory
behind the KASUMI cipher (chapter \ref{ch:kas}) and the different TMTO
attacks (chapter \ref{ch:tmto}). We will proceed to look into the
actual table and parameter choices for the TMTO-attack (chapter
\ref{ch:param}). Then we will go in depth with the
implementation of both KASUMI and the TMTO route we took(chapter
\ref{ch:impl}). Next we will analyze the results of the attack
(chapter \ref{ch:anal}). Afterwards we will move on to a discussion
about what further improvements could be made (chapter
\ref{ch:disc}). Finally we will give a conclusion of the project
(chapter \ref{ch:concl}).

%%% Local Variables:
%%% mode: latex
%%% TeX-master: "Thesis"
%%% End:
