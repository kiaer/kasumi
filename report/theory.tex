\chapter{Theory}

\section{Kasumi}

The KASUMI block cipher is used in the GSM 
\subsection{Key schedule}
The keyschedule of the Kasumi cipher calculates all of the round keys,
from the given input key. The 128-bit key is split into eight 16-bit
subkeys, where $K_1,..K_8$ is a concatenation of subsequent 16-bit of
the original 128-bit key $K$.
\[K = K_1 || K_2 || K_3 || K_4 || K_5 || K_6 || K_7 || K_8\]
A modified key $K'$ is also generated by XOR'ing the original key with
the number $0x123456789ABCDEFFEDCBA9876543210$. This key $K'$ is split
into eight 16-bit subkeys as well, following the same rules of the
original key.
\[K' = K'_1 || K'_2 || K'_3 || K'_4 || K'_5 || K'_6 || K'_7 || K'_8\]
Thereafter eight round keys are generated as follows:
\begin{align*}
  KLi1 &= ROL16(K_i,1)\\
  KLi2 &= K'_{i+2}\\
  KOi1 &= ROL16(K_{i + 1},5)\\
  KOi2 &= ROL16(K_{i + 5},8)\\
  KOi3 &= ROL16(K_{i + 6},13)\\
  KIi1 &= K'_{i+4}\\
  KIi2 &= K'_{i+3}\\
  KIi3 &= K'_{i+7}\\
\end{align*}

\subsection{Algorithm}
The Kasumi algorithm takes an 64-bit word as an input. This word is
split into two halves.
\[ inputword = R_o || L_o\]
In each round of the algorithm the right half of the input is XOR'ed
with the output of the round functions. Thereafter the right and left
values are swapped. 
\begin{align*}
  L_i &= F_i(KL_i,KO_i,KI_i,L_{i - 1}) \oplus R_{i - 1} \\
  R_i &= L_{i - 1}
\end{align*}
Where $i$ determines which round function is used. A different round
function is used for odd and even rounds of the algorithm.

An odd round consists of the following:
\[F_{i = odd} = FO(KO_i, KI_i, FL(KL_i, L_{i - 1})) \]
Whereas an even round is performed as follows:
\[F_{i = even} = FL(KL_i, FO(KO_i,KI_i, L_{i - 1})) \]
After the last round the output ciphertext is the concatenation of the
outputs.
\[cipher = R_8 || L_8\]

#TODO
FL OG FO OG FI

\section{TMTO-Attack}



\subsection{Hellman Tables}

\subsection{Distinguished Points Tables}

\subsection{Rainbow Tables}

%%% Local Variables:
%%% mode: latex
%%% TeX-master: "Thesis"
%%% End:
