\chapter{Implementation}
\section{KASUMI Cipher}
\subsection{Implementation}
The implemenation of the KASUMI follows the theory described in the
chapter \nameref{ch:kas}.


\subsection{Benchmarks and Performance}
The benchmarks of the KASUMI cipher implemented is performed on two
laptops. The Laptops will each compute the encryption on 64-bit of
data 10.000.000 times. An average time of ten runs of this test is
chosen as the result. Both machines run on Intels Hashwell
architecture with Core i7 CPUs. The specifications of the laptops are as follows:
\begin{table}[h!]
    \begin{tabular}{l|l|l}
                                    & Zenbook - i7           & Yoga
                                                               pro 2 -
      i7\\ \hline
    CPU Frequency                   & 1,8 GHZ @ 2.9ghz TURBO & 2,0 GHZ @ 3.0ghz TURBO \\ \hline
    CPU Cycles/s                    & 2900000000             & 3000000000             \\ \hline
    Times encryption of 64 bit data & 10000000               & 10000000               \\ \hline
    Total bit encyrpted             & 640000000              & 640000000              \\
    \end{tabular}
    \caption{CPU Specs of laptops performing benchmarks}
    \label{tab:specs}
\end{table}\\
The tests will be performed with the
GCC-compiler \footnote{The Gnu Compiler Collection -
  https://gcc.gnu.org/}. They will be performed with different compile
flags for optimization, consisting of no flags, O2, O3 and Ofast as
these are the most common flags for optimization. The two tables
\ref{tab:zen} and \ref{tab:yoga} will show the results of the KASUMI
tests.
\begin{table}[h!]
    \begin{tabular}{l|l|l|l|l}
     Zenbook 1.8 ghz.  & ~                     & ~             & ~              & ~               \\
    GCC compile flags. & Time in sec (average) & Cycles in tot & Cycles per bit & Cycles per byte \\ \hline
    None               & 7,2322                & 20973380000   & 32,77090625    & 262,16725       \\ \hline
    O2                 & 2,0035                & 5810150000    & 9,078359375    & 72,626875       \\ \hline
    O3                 & 1,8945                & 5494050000    & 8,584453125    & 68,675625       \\ \hline
    Ofast              & 1,897                 & 5501300000    & 8,59578125     & 68,76625        \\
    \end{tabular}
    \caption{Zenbook i7 benchmarks}
    \label{tab:zen}
\end{table}
\begin{table}[h!]
    \begin{tabular}{l|l|l|l|l}
     Yoga 2 pro 2.0 ghz. & ~                     & ~             & ~              & ~               \\
    GCC compile flags.   & Time in sec (average) & Cycles in tot & Cycles per bit & Cycles per byte \\ \hline
    None                 & 6,885                 & 20655000000   & 32,2734375     & 258,1875        \\ \hline
    O2                   & 1,933636364           & 5800909091    & 9,063920455    & 72,51136364     \\ \hline
    O3                   & 1,823333333           & 5470000000    & 8,546875       & 68,375          \\ \hline
    Ofast                & 1,835                 & 5505000000    & 8,6015625      & 68,8125         \\
    \end{tabular}
    \caption{Yoga 2 i7 benchmarks}
    \label{tab:yoga}
\end{table}\\


For analysis of the performance of the cipher, gprof \footnote{GNU
  Profiler - https://sourceware.org/binutils/docs/gprof/} is used to
analyze each function and get a clear idea of which functions might
cause slowdowns. The following output is produced from running the
tests and analyzing with gprof:
\begin{lstlisting}[caption=Gprof output,captionpos=b,label=lst:grpof]
    %   cumulative   self              self     total           
 time   seconds   seconds    calls  ns/call  ns/call  name    
 38.56      0.53     0.53 240000024     2.20     2.20  fi
 23.50      0.85     0.32                             keyschedule
 18.36      1.10     0.25 80000008     3.14     9.75  fo
 12.49      1.27     0.17 80000008     2.14     2.14  fl
  4.41      1.33     0.06                             kasumi_enc
\end{lstlisting}
Note that for getting an somewhat accurate representation of the time usage of
each function, the functions of the cipher cannot be inlined. Thus the
attribute \textit{noinline} must be included for each function. This
takes away some of the optimzations done by the Ofast compile
flag. Not supringsingly it shows that \textit{fi} and
\textit{keyschedule} takes up the most computational
power. Since \textit{fi} is used in every round of the cipher and it
requires lookups in a predefined table for every computation, this
could be seen as expected. As for the keyschedule, it's recalculated
in each encryption and therefore will require a new computation in
each new encryption done. 

If we consider the results of \ref{lst:grpof} and look at both
\ref{tab:zen} and \ref{tab:yoga}, this shows us that for each byte
encrypted around 17 cycles of the CPU is used on generating the
keyschedule. That leaves around 50 cylces remaining for the actual
encryption of a text.

Since most of the actual computations TODO assembly vs C? Find kilder?

Further optimization could be gained by using a different compiler. As
both test machines contain Intels i7 CPUs, noticeable performance
increases could be gained by using a Intel compiler. Tests with the
Intel compiler has not been performed. Kilder?

\section{Tablegenerator}

\section{Online Phase}

\section{Usage}


%%% Local Variables:
%%% mode: latex
%%% TeX-master: "Thesis"
%%% End:
