\chapter{Implementation}
\section{KASUMI Cipher}
\subsection{Implementation}
\subsection{Benchmarks}
The benchmarks of the KASUMI cipher implemented is performed on two
laptops. The Laptops will each compute the encryption on 64-bit of
data 10.000.000 times. An average time of ten runs of this test is
chosen as the result. Both machines run on Intels Hashwell
architecture with Core i7 CPUs. The specifications of the laptops are as follows:
\begin{table}[h!]
    \begin{tabular}{l|l|l}
                                    & Zenbook - i7           & Yoga
                                                               pro 2 -
      i7\\ \hline
    CPU Frequency                   & 1,8 GHZ @ 2.9ghz TURBO & 2,0 GHZ @ 3.0ghz TURBO \\ \hline
    CPU Cycles/s                    & 2900000000             & 3000000000             \\ \hline
    Times encryption of 64 bit data & 10000000               & 10000000               \\ \hline
    Total bit encyrpted             & 640000000              & 640000000              \\
    \end{tabular}
    \caption{CPU Specs of laptops performing benchmarks}
    \label{tab:specs}
\end{table}\\
The tests will be performed with the
GCC-compiler \footnote{The Gnu Compiler Collection -
  https://gcc.gnu.org/}. They will be performed with different compile
flags for optimization, consisting of no flags, O2, O3 and Ofast as
these are the most common flags for optimization. 
\begin{table}[h!]
    \begin{tabular}{l|l|l|l|l}
     Zenbook 1.8 ghz.  & ~                     & ~             & ~              & ~               \\
    GCC compile flags. & Time in sec (average) & Cycles in tot & Cycles per bit & Cycles per byte \\ \hline
    None               & 7,2322                & 20973380000   & 32,77090625    & 262,16725       \\ \hline
    O2                 & 2,0035                & 5810150000    & 9,078359375    & 72,626875       \\ \hline
    O3                 & 1,8945                & 5494050000    & 8,584453125    & 68,675625       \\ \hline
    Ofast              & 1,897                 & 5501300000    & 8,59578125     & 68,76625        \\
    \end{tabular}
    \caption{Zenbook i7 benchmarks}
    \label{tab:zen}
\end{table}\\
asd
\begin{table}[h!]
    \begin{tabular}{l|l|l|l|l}
     Yoga 2 pro 2.0 ghz. & ~                     & ~             & ~              & ~               \\
    GCC compile flags.   & Time in sec (average) & Cycles in tot & Cycles per bit & Cycles per byte \\ \hline
    None                 & 6,885                 & 20655000000   & 32,2734375     & 258,1875        \\ \hline
    O2                   & 1,933636364           & 5800909091    & 9,063920455    & 72,51136364     \\ \hline
    O3                   & 1,823333333           & 5470000000    & 8,546875       & 68,375          \\ \hline
    Ofast                & 1,835                 & 5505000000    & 8,6015625      & 68,8125         \\
    \end{tabular}
    \caption{Yoga 2 i7 benchmarks}
    \label{tab:yoga}
\end{table}\\

Further optimization could be gained by using a different compiler. As
both test machines contain Intels i7 CPUs, noticeable performance
increases could be gained by using a Intel compiler.


\section{Tablegenerator}

\section{Online Phase}

\section{Usage}


%%% Local Variables:
%%% mode: latex
%%% TeX-master: "Thesis"
%%% End:
