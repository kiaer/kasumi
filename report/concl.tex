\chapter{Conclusion}
\label{ch:concl}
% A summary of the main part of the text
% A deduction made on the basis of the main body
% Your personal opinion on what has been discussed
% A statement about the limitations of the work
% A comment about the future based on what has been discussed
% The implications of the work for future research

In this project we tried to analyze a practical attack on the KASUMI
block cipher by implementing a time-memory trade-off attack. We
conducted experiments to determine whether or not this attack can be
seen as practical. 

The biggest obstacle in performing this attack will undoubtedly be
computation of the tables needed for the attack. Generating such table
will require more than $2^{64}$ computations of KASUMI if a decent
success probability is expected.

Our implementation of the attack is also very unoptimized when
considering large tables consisting of multiple terabytes of
data. Read times and lookups in tables will prove to be a huge
hindrance if the attack is to be performed on the full keyspace of
\code{64-bit}.

But even then, when compared to previous attacks, the online time
required for the attack matches up and even exceeds most of the
previous attacks performed on KASUMI. This is done attacking the 2G
compatible version of KASUMI, but nonetheless it attacks the full key
and every round key.

If the attack is to be considered doable the combined approach of 
rainbow and DP attacks should definitely be a part of the
implementation. Additional storage improvements might also be needed
to reach optimal lookup times.

An approach where GPUs is used for parallelization is also something
that should be explored in more depth, as these contains computing
power exceeding the CPUs at a much cheaper price.

If these points gets looked into, we can look at an actual attack on
KASUMI that can be performed in days using a relatively cheap
setup, assuming a rainbow table is computed.


